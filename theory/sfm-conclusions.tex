\section{Выводы}

В этой главе был рассмотрен метод построения 3D реконструкции из набора снимков, называемый Structure from Motion. Были подробно разобраны составляющие его этапы и проанализированы различные алгоритмы, которые могут использоваться на каждом шаге метода. Среди рассмотренных дескрипторов и детекторов можно выделить SIFT - как самый точный и надёжный, и современный ORB - как быстрый, но в то же время достаточно устойчивый.

Как показали практические эксперименты Structure from Motion больше применим для офлайн построения больших карт местности и 3D реконструкций, чем для работы в реальном времени на борту. Он точный, но медленный. Так как для успешной навигации требуется онлайн работа, то это приводит к необходимости проведения дальнейших исследований и поиска подходящего метода. Метод SLAM будет рассмотрен в следующей главе.