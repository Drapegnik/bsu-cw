\chapter{ОБЛАСТЬ КОМПЬЮТЕРНОГО ЗРЕНИЯ}

\section{Актуальность и практическая значимость}

Как следует из названия БПЛА не имеют пилота, но это не значит, что они не пилотируемы. Управление беспилотником требует специального обучения и сосредоточенности, является очень утомительным для оператора. Основополагающим необходимым условием для работы дрона является наличие GPS сигнала, что делает его очень уязвимым и зависимым от внешних обстоятельств. В отсутствие сигнала системы глобального позиционирования дрон теряет управление.

В связи с этим возникает задача нахождения и использования альтернативных источников навигации. Так как почти каждый современный беспилотник оснащён камерой, то возможно применение алгоритмов компьютерного зрения для решения данной задачи.

С помощью таких алгоритмов и программного обеспечения должна быть возможна навигация дрона, используя только камеру. Дополнительные возможности применения обширны: патрулирование заданной территории и обнаружение новых объектов, не присутствовавших ранее, возвращение в заданную точку в случае потери gps сигнала, слежение за данным объектом, построение 3D карт местности, навигация по заданной цифровой карте.

\section{Постановка задачи}

Необходимо изучить и проанализировать различные существующие алгоритмы компьютерного зрения, провести практические эксперименты и сравнения, впоследствии применить накопленные знания для разработки алгоритма решения задачи навигации беспилотного летательного аппарата в условиях отсутствия GPS сигнала, с использованием имеющихся на борту оптических приборов. Требуется изучить и разобраться в принципах работы Robot Operating System (ROS). Ниже будут рассмотрены такие глобальные подходы как Structure from Motion (SFM) и Simultaneous Localization and Mapping (SLAM), с использованием таких алгоритмов как Bundle Adjustment, SIFT, SURF, ORB и их различных реализаций.

\section{Основные теоретические понятия}

Люди с рождения обладают зрением, что позволяет нам распознавать изображения и объекты на них, сравнивать их между собой, оценивать расстояния и размеры. Всё это человеческий мозг делает бессознательно, автоматически. Однако, для машины изображение — всего лишь закодированные данные, набор нулей и единиц. Одной из основных проблем в сопоставлении изображений является очень большая размерность пространства, которое обладает информацией. Если взять маленькую картинку размером хотя бы $100*100$ пикселей, то уже получим количество информации равное $10^4$ пикселей. Поэтому методы анализа изображения должны быть быстрыми и эффективными.

Так как же машина обретает зрение?

Основная идея состоит в том, чтобы получить какую-то характеристику, которая будет хорошо описывать изображение, легко вычисляться и для которой можно ввести оператор сравнения. Эта \quotes{характеристика} должна быть устойчива к различным преобразованиям (сдвиг, поворот и масштабирование изображения, изменение яркости, изменение положения камеры). Это необходимо для того, чтобы было возможно определить один и тот же объект на изображениях сделанных с разных углов, расстояний и при разном освещении.

Все эти условия приводят к необходимости выделения на изображении особых, \textit{ключевых точек} (\textbf{key points}). Этот процесс называется \textit{извлечение признаков} (\textbf{feature extraction}). Ключевая точка - это такая особая точка, которая достаточно хорошо отличается от близлежащих точек по какой-то определённой характеристике. Она должна быть сильно \quotes{не похожа} на остальные точки вокруг, соответственно может являться, в какой-то степени, уникальным свойством изображения в своей локальной области. Таким образом машина может представить изображение как модель, состоящую из особых точек. Например, на изображении человеческого лица функции ключевых точек могут выполнять глаза, уголки губ, кончик носа.

После выделения особых точек компьютеру нужно уметь их сравнивать (отличать друг от друга). Этот процесс называется \textit{сопоставление признаков} (\textbf{feature matching}). Для сравнения удобно использовать \textit{дескрипторы} (\textbf{descriptor} - \quotes{описатель}). Дескриптор - своеобразный идентификатор ключевой точки, представляющий её в удобном для сравнения и понятном для машины виде. Как мы увидим далее, именно благодаря дескрипторам получается инвариантность относительно преобразований изображений.
