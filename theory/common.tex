\section{Постановка задачи}

Необходимо изучить и проанализировать различные существующие алгоритмы компьютерного зрения, провести практические эксперименты и впоследствии применить накопленные знания для решения задачи навигации беспилотного летательного аппарата в условиях отсутствия GPS сигнала, с использованием имеющихся с оптических приборов. Ниже будут рассмотрены такие глобальные подходы как Structure from Motion (SFM) и Simultaneous Localization and Mapping (SLAM), с использованием таких алгоритмов как Bundle Adjustment, SIFT, SURF, ORB и их различных реализаций.

\section{Общие теоретические положения}

Люди обладают зрением, что позволяет нам распознавать изображения и объекты на них, сравнивать их между собой и всё это мы делаем бессознательно, автоматически. Однако, для машины изображение — всего лишь закодированные данные, набор нулей и единиц. Одной из больших проблем в сопоставлении изображений является очень большая размерность пространства, которое несёт информацию. Если взять картинку размером хотя бы $100*100$ пикселей, то уже получим размерность равную $10^4$ пикселей. Поэтому методы анализа изображения должны быть быстрыми и эффективными.

Как же компьютер обретает зрение?

Основная идея состоит в том, чтобы получить какую-то характеристику, которая будет хорошо описывать изображение, легко вычисляться и для которой можно ввести оператор сравнения. Эта \quotes{характеристика} должна быть устойчива к различным преобразованиям (сдвиг, поворот и масштабирование изображения, изменения яркости, изменения положения камеры). Чтобы определять один и тот же объект на изображениях сделанных с разных углов, расстояний и при разном освещении.

Все эти условия приводят к необходимости выделения на изображении особых, \textit{ключевых точек} (\textbf{key points}). Этот процесс называется \textit{извлечение признаков} (\textbf{feature extraction}). Ключевая точка - это такая особая точка, которая сильно отличается от близлежащих точек по какой-то обусловленной характеристике. Она должна быть не похожа на остальные точки вокруг, соответственно является, в какой-то степени, уникальным свойством изображения в своей локальной области. Таким образом машина может представить изображение как модель, состоящую из особых точек. Например, на изображении человеческого лица функции ключевых точек могут выполнять глаза, уголки губ, кончик носа.

После выделения особых точек компьютеру нужно уметь их сравнивать (отличать друг от друга). Этот процесс называется \textit{сопоставление признаков} (\textbf{feature matching}). Для сравнения удобно использовать \textit{дескрипторы} (\textbf{descriptor} - \quotes{описатель}). Дескриптор - своеобразный идентификатор ключевой точки, представляющий её в удобном для сравнения виде. Как мы увидим далее, именно благодаря дескрипторам получается инвариантность относительно преобразований изображений.
