\section{Сравнение различных SLAM-алгоритмов}

Все рассмотренные в данном разделе алгоритмы и их реализации - находятся в открытом доступе и их исходный код можно найти на GitHub.

\subsection{Parallel Tracking And Mapping}
\textbf{Parallel Tracking And Mapping} (\hyperref[itm:ptam]{PTAM [\ref{itm:ptam}]}) - визуальный метод адаптирующий SLAM для для области дополненной реальности.

Впервые в PTAM (в 2007 году) было предложено распараллелить процессы локализации и построения карты. Является прародителем современных быстрых SLAM решений. Как и большинство визуальных методов использует Bundle Adjustment и хранит карту как облако точек. Подробнее работа алгоритма была рассмотрена в предыдущей секции.

\subsection{Large Scale Direct SLAM}
\textbf{Large Scale Direct SLAM} (\hyperref[itm:lsd]{LSD-SLAM [\ref{itm:lsd}]}) - плотный прямой монокулярный SLAM метод, который вместо использования ключевых точек оперирует непосредственно интенсивностью (числовыми значениями пикселей) изображения.

LSD-SLAM параллельно запускает три функции: трекинг (локализация), построение карты и оптимизация карты. В отличие от методов основанных на ключевых точках, прямой прямой метод не представляет изображения в виде абстракции (набора ключевых точек), а сохраняет полные изображения.

Вместо минимизации ошибки проекций точек в этом прямом методе минимизируется попиксельная разность (фотометрическая разность) на ключевых кадрах. Для построения и уточнения карты оценивается попиксельная глубина, вместо оценки особенностей изображений, таких как 3d точки и векторы нормали.

Также в прямых методах, за счёт сохранения большего количества точек, получается гораздо более плотная и полная 3D реконструкция.

LSD-SLAM работает в реальном времени даже на CPU, что позволяет использовать его даже на современных смартфонах. Также он является монокулярным - использует только одну камеру. Поэтому он получил широкое распространение в сфере дополненной реальности (\textit{augmented reality}).

\subsection{ORB SLAM}
\textbf{ORB SLAM} - \hyperref[itm:orbslam]{опубликованный в 2015 [\ref{itm:orbslam}]} монокулярный метод, основанный на особых точках. Как следует из названия алгоритм использует для извлечения ключевых точек рассмотренный ранее детектор \hyperref[itm:orb]{ORB (Oriented FAST and Rotated BRIEF)[\ref{itm:orb}]}. Благодаря высокой скорости ORB (извлечение особых точек за милисекунды) достигается работа в реальном времени, а инвариантность ORB к поворотам камеры и смене освещения обеспечивает надёжность метода.

В настоящее время метод называют лучшим из всех SLAM основанных на особых точках.

\subsection{Semi-direct Visual Odometry}
Выбирая между прямыми и основанными на особых точках методами некоторые разработчики решили извлечь все лучшие стороны из обоих подходов.

\textbf{Semidirect
Visual Odometry} (\hyperref[itm:svo]{SVO [\ref{itm:svo}]}) - полупрямой SLAM метод. Карта в этом подходе представляет собой координаты FAST (Features from Accelerated Segment Test) особых точек. С другой стороны, как и в прямых методах, для оценки используется минимизация разницы глубин, поэтому FAST особенности помещаются в глубинный фильтр и позднее используются для такой оценки.
SVO разрабатывался специально для работы на борту микро летательных аппаратов. Как утверждают разработчики алгоритм работает в реальном времени на скорости 55 кадров в секунду на встроенном бортовом компьютере и 300 кадров в секунду на обычном ноутбуке.