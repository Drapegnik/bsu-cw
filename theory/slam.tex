\chapter{МЕТОД SLAM}

\section{Описание метода}

\textbf{Simultaneous Localization and Mapping (SLAM)} -- метод одновременной локализации и построения карты, первоначально предложенный Питером Чесманом (\textit{Peter Cheeseman}) и Рэндалом Смитом (\textit{Randall C. Smith}) для задачи ориентации колёсных роботов с лазерными сенсорами в плоских пространствах и опубликованный в 1986 году \hyperref[itm:cheeseman]{[\ref{itm:cheeseman}]}.

В общем случае SLAM решает проблему построения или обновления карты неизвестной окружающей среды и одновременной навигации по ней. Используется в различных средах и с различными роботизированными устройствами, например, self-driving автомобили, домашние роботы-пылесосы, планетоходы, воздушные и даже подводные беспилотные аппараты.

Существует много различных реализаций и вариаций данного метода. Это связано с тем фактором, что алгоритм сильно зависит от окружающей среды, в которой предполагается его использование. Так, например, выделяют среды с многочисленными ярко выраженными ориентирами (ландшафты с отдельно стоящими деревьями) и с их отсутствием (коридоры, комнаты). Так же алгоритмы разделяют по типу строящейся карты -- плоская или трёхмерная. В первом случае может строиться карта препятствий -- массив, где элементы, отражающие положение препятствий, имеют значение 1, а все остальные — 0, а во втором -- облако точек, описывающее окружающий мир.

SLAM представляет \quotes{проблему курицы и яйца}: карта нужна для навигации по ней, но в то же время оценка текущего положения нужна для построения карты. Несмотря на это существуют алгоритмы, решающие эту проблему в реальном времени.

Метод SLAM можно разбить на этапы:
\begin{enumerate}
    \item Извлечение ориентиров (\textit{Landmarks extraction});
    \item Объединение (сопостовление) ориентиров с разных позиций (\textit{Data association});
    \item Оценка положения (\textit{State estimation});
    \item Обновление положения и ориентиров (\textit{State and landmarks update}).
\end{enumerate}

Одной из важнейших составляющих SLAM процесса, является получение сведений о текущей позиции робота через одометрию. Одометрия -- использование данных о движении приводов, углах поворота колёс/лопастей, текущей скорости, показаний гироскопа для оценки перемещения. Одометрия даёт лишь приблизительное положение, которое SLAM уточняет, основываясь на ориентирах.

Следующие требования предъявляются к ориентирам:
\begin{enumerate}
    \item Они должны легко выделяться с разных позиций используемым средством восприятия окружающей среды;
    \item Они должны быть уникальны и легко отличимы друг от друга;
    \item Их должно быть достаточное количество в окружающей среде;
    \item Они должны быть стационарными.
\end{enumerate}

При использовании для навигации БПЛА камеры под все описанные требования подходят ключевые точки, которые мы рассматривали ранее. Они могут быть выделены и сопоставлены с помощью уже рассмотренных алгоритмов SIFT или ORB. В таком случае SLAM называют основанным на особых точках (\textit{feature based}).

Рассмотрим работу алгоритма на примере:
\begin{enumerate}
    \item Робот выделяет ориентиры и их позиции;
    \item Робот двигается. Через одометрию он вычисляет позицию, в которой он \quotes{думает}, что находится;
    \item Робот опять оценивает положения ориентиров, но получает, что они находятся не там, где он \quotes{думал} на предыдущим шаге. Это означает, что положение, полученное через одометрию, не является точным;
    \item Через данные о действительном расположении ориентиров обновляется позиция робота, то есть происходит уточнение;
    \item Положения ориентиров сохраняются для дальнейшей навигации по ним, алгоритм переходит на первый шаг.
\end{enumerate}

Для возможности работы алгоритма в реальном времени обработка каждой следующей порции данных (ориентиров, кадров с камеры) должна выполнятся до того, как будут получены новые данные. Это и отличает SLAM от рассмотренного ранее SFM, где на вход поступает сразу большое количество изображений и долго обрабатываются в режиме офлайн.

Для триангуляции и расположения камер используется рассмотренный ранее Bundle Adjustment. Проблема в том что время его работы быстро растёт с увеличением размера построенной карты. Решается данная проблема с помощью разнесения задач локализации и построения карты по разным процессам. Ориентация происходит в реальном времени, а Bundle adjustment перестраивает карту в фоновом режиме. По завершению работы фоновый процесс обновляет карту для процесса ориентации, а тот, в свою очередь, добавляет новые ориентиры для уточнения карты и это повторяется бесконечно.
