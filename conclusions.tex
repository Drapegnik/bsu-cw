\chapter*{ЗАКЛЮЧЕНИЕ}
\addcontentsline{toc}{chapter}{ЗАКЛЮЧЕНИЕ}

Навигация беспилотных летательных аппаратов -- сложная и актуальная задача, требующая применения различных подходов. Компьютерное зрение -- очень прогрессивная область информатики, которая получает применение в разнообразных сферах, в особенности автоматизации человеческого труда.

В процессе выполнения данной работы мной:

\begin{itemize}
    \item рассмотрены основные алгоритмы компьютерного зрения;
    \item проанализированы дескрипторы SIFT, SURF, ORB и проведены сравнительные эксперименты;
    \item детально разобрана и настроена операционная система для программирования роботов ROS;
    \item предложен алгоритм восстановления местоположения по 3D карте местности;
    \item спроектировано и разработано приложение для построения и визуализации 3D карты местности, а также поиска по ней с использованием таких методов как Structure From Motion и Bundle Adjustment;
    \item с помощью ROS продемонстрирован пример совместной работы алгоритма Simultaneous localization and mapping и разработанного приложения в реальном времени с использованием веб-камеры.
\end{itemize}

В результате исследований и экспериментов выявлено, что последние реализации алгоритма SLAM, адаптированные для работы с камерами и на небольших мобильных устройствах, являются идеальными кандидатами для решения задачи навигации по 3D карте местности в режиме реального времени. Следовательно они могут быть использованы на борту беспилотного летательного аппарата в условиях отсутствия GPS сигнала. Использование результатов работы SLAM алгоритма в качестве начальных данных для процесса SFM позволяет быстрее получить детальную карту местности.

\newpage
