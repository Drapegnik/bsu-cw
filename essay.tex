\begin{center}
  \large\bfseries{РЕФЕРАТ}
\end{center}

Дипломная работа, \pages страницы, \figures рисунков, \sources источников.

\begin{justify}
    \large{НАВИГАЦИЯ, БПЛА, РЕКОНСТРУКЦИЯ, КОМПЬЮТЕРНОЕ ЗРЕНИЕ, ОСОБЫЕ ТОЧКИ, ДЕСКРИПТОРЫ, ДЕТЕКТОРЫ, СОПОСТАВЛЕНИЕ ИЗОБРАЖЕНИЙ}
\end{justify}

Объектом исследования являются способы навигации беспилотных летательных аппаратов и алгоритмы компьютерного зрения.

\vspace{1em}

Цель работы -- изучение возможности применения методов компьютерного зрения в системах навигации беспилотных летательных аппаратах.

\vspace{1em}

Методы исследования: изучение соответствующей литературы и публикаций, проведение экспериментов, разработка программного обеспечения.

\vspace{1em}

В результате работы были рассмотрены и проанализированы различные существующие алгоритмы компьютерного зрения, предложен алгоритм восстановления местоположения по цифровой карте местности, разработано приложение для построения цифровой карты местности и осуществления поиска по ней. Проведены сравнительные эксперименты.

\vspace{1em}

Области применения: системы навигации, модели и алгоритмы, работающие на борту беспилотных летательных аппаратов, реконструкция 3D карты местности.
\newpage



\begin{center}
     \large\bfseries{РЭФЕРАТ}
\end{center}

Дыпломная праца, \pages старонкі, \figures малюнкаў, \sources крыніц.

\begin{justify}
  \large{НАВІГАЦЫЯ, БПЛА, РЭКАНСТРУКЦЫЯ, КАМПУТАРНЫ ЗРОК, КЛЮЧАВЫЯ КРОПКІ, ДЭСКРЫПТАРЫ, ДЭТЭКТАРЫ, СУПАСТАЎЛЕННЕ ВЫЯВАЎ}
\end{justify}

Аб'ектам даследвання з'яўляюцца метады навігацыі беспілотных лятаючых апаратаў і алгарытмы кампутарнага зроку.

\vspace{1em}

Мэта працы -- даследванне магчымасці выкарыстоўвання метадаў кампутарнага зроку ў сістэмах навігацыі беспілотных лятаючых апаратах

\vspace{1em}

Метады даследвання: аналіз адпаведнай літаратуры і публікацый, правядзенне эксперыментаў, распрацоўка праграмнага забеспячэння.

\vspace{1em}

У выніку працы былі разгледжаны і прааналізаваны разнастайныя існуючыя алгарытмы кампутарнага зроку, прапанаваны алгарытм аднаўлення месцазнаходжання па 3D мапе мясцовасці, распрацавана прыкладанне для пабудовы 3D мапы мясцовасці і ажыццяўлення пошука па ёй. Праведзены параўнаўчыя эксперыменты.

\vspace{1em}


Галіны прымянення: сістэмы навігацыі, мадэлі і алгарытмы, якія працуюць на барту беспілотных лятаючых апаратаў, рэканструкцыя 3D мапы мясцовасці.
\newpage



\begin{center}
     \large\bfseries{ABSTRACT}
\end{center}

Graduate work, \pages pages, \figures pictures, \sources sources.

\begin{justify}
  \large{NAVIGATION, UAV, RECONSTRUCTION, COMPUTER VISION, KEYPOINTS, DESCRIPTORS, DETECTORS, FEATURE MATCHING}
\end{justify}

The object of the research are methods of navigation of unmanned aerial vehicles and computer vision algorithms.

\vspace{1em}


Purpose -- to research the possibility of applying computer vision algorithms in the navigation systems of unmanned aerial vehicles.

\vspace{1em}

Methods of the research are: to analyze publications, to perform experiments, to develop the software.

\vspace{1em}

As a result of the work, various existing computer vision algorithms were examined and analyzed, an algorithm for restoring the location on a digital terrain map was proposed, an application for constructing a digital terrain map and searching for it was developed. Comparative experiments were carried out.

\vspace{1em}


The scopes are: navigation systems, models and algorithms working on board unmanned aerial vehicles, reconstruction of a 3D map.
\newpage