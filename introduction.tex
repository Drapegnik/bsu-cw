\chapter*{ВВЕДЕНИЕ}
\addcontentsline{toc}{chapter}{ВВЕДЕНИЕ}

В настоящее время интенсивно развивается такая область информатики как компьютерное зрение (computer vision). Существует множество алгоритмов по распознаванию и поиску объектов на картинке, сравнения изображений, определения того, как с помощью геометрический преобразований и/или масштабирования можно из одного изображения получить другое. Самая популярная библиотека, которая предоставляет реализации основных алгоритмов - решение с открытым исходным кодом \hyperref[itm:opencv]{OpenCV [\ref{itm:opencv}]}.

Алгоритмы компьютерного зрения активно используются в системах управления процессами (промышленные роботы, автономные транспортные средства), системах видеонаблюдения, системах организации информации (индексация баз данных изображений), системах моделирования объектов или окружающей среды (анализ медицинских изображений, топографическое моделирование), системах взаимодействия (устройства ввода для системы человеко-машинного взаимодействия), системы дополненной реальности.

Крупнейшая мировая IT корпорация Google разрабатывает self-driving cars (машины с автопилотом) и предполагается, что в будущем человеку вообще не придётся управлять автомобилем. Это должно уменьшить число происшествий исключая \quotes{человеческий фактор} и, соответственно, сделать передвижение с помощью автомобиля безопаснее. Самый популярный сервис такси - Uber уже использует машины с автопилотом, что в будущем позволит снизить стоимость услуг сокращением траты средств на человеческие ресурсы (Компания уже уменьшила траты, используя мобильное приложение вместо диспетчеров). Американская компания Amazon открыла магазин без кассиров, в котором с помощью алгоритмов компьютерного зрения определяется какие товары клиент положил себе в корзину и их стоимость автоматически списывается с карты при выходе из магазина.

Таким образом компьютерное зрение, наряду с машинным обучением, является сейчас наиболее новой и активно развивающейся областью информатики, используемой всеми лидерами отрасли. Основное применение компьютерного зрения - уменьшение человеческой работы, высвобождения одного из самых дорогих ресурсов - человеческого времени.

\newpage