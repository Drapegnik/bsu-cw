\chapter{АКТУАЛЬНОСТЬ ЗАДАЧИ}

\section{Навигация с помощью GPS}
GPS (Global Positioning System) -- спутниковая система радио-навигации, которая позволяет, с помощью GPS приёмника, определять координаты и время в любой точке Земли, из которой беспрепятственно видны четыре или более GPS спутника. Система разработана в Соединённых Штатах Америки для нужд  Военно-воздушных сил, в 1980х годах доступ к системе был предоставлен для гражданских лиц. Так как система принадлежит правительству США и полностью подконтрольна ему, существует возможность для выборочного ограничивания доступа к системе или ограничение качества сигнала, которая и была использована в 1999 году против индийской армии в ходе Каргильской войны. Также такие препятствия как здания или горы ослабляют сигнал.

Основная идея, которая лежит в основе работы системы: зная точные координаты спутников в определённый момент времени, благодаря эффекту Доплера (изменения частоты принимаемого сигнала при изменении положения передатчика), можно вычислить текущее положение относительно передатчиков (спутников). На GPS спутниках установлены очень точные атомные часы которые синхронизированы между всеми спутниками и земным временем. Также с большой точностью известны положения каждого спутника. Спутники непрерывно транслируют данные о своём текущем положении и времени на борту. GPS приёмник принимает эти сигналы от четырёх спутников и решает систему уравнений с четырьмя неизвестными: тремя координатами и временем. Точность такого метода составляет до пяти метров.

\section{Актуальность и практическая значимость}

Как следует из названия, беспилотные летательные аппараты (дроны) не имеют пилота, но это не значит, что они не пилотируемы. Управление беспилотником требует специального обучения и сосредоточенности, является очень утомительным для оператора. Основополагающим условием для работы дрона является наличие GPS сигнала, что делает его очень уязвимым и зависимым от внешних обстоятельств. В отсутствие сигнала системы глобального позиционирования дрон теряет управление.

GPS не очень надёжная система, в работе которой могут быть сбои и помехи. В связи с этим возникает задача нахождения и использования альтернативных источников навигации. Так как почти каждый современный беспилотник оснащён камерой, то возможно применение алгоритмов компьютерного зрения для решения данной задачи. С помощью таких алгоритмов и программного обеспечения должна быть возможна навигация дрона, используя только камеру. Зная начальное местоположение и отслеживая с помощью алгоритмов компьютерного зрения свое перемещение, робот может вычислять текущее местоположение относительно начального.

Дополнительные возможности применения обширны: патрулирование заданной территории и обнаружение новых объектов, не присутствовавших ранее (например обнаружение лесных пожаров), возвращение в заданную точку в случае потери GPS сигнала, слежение за данным объектом, построение 3D карт местности, навигация по заданной цифровой карте.

\section{Постановка задачи}

Цель -- исследовать возможность применения камеры дрона для определения текущего местоположения и решения задачи навигации беспилотного летательного аппарата в условиях отсутствия GPS сигнала.

Для достижения цели необходимо:
\begin{enumerate}
    \item Изучить и проанализировать существующие алгоритмы компьютерного зрения:
        \begin{itemize}
            \item дескрипторы SIFT, SURF, ORB;
            \item алгоритм проективной геометрии Bundle Adjustment;
            \item метод Structure from Motion (SFM);
            \item метод Simultaneous Localization and Mapping (SLAM).
        \end{itemize}
    \item Провести практические эксперименты и сравнения разных подходов;
    \item Разобраться в принципах работы операционной системы Robot Operating System (ROS);
    \item Реализовать приложение для построения 3D карты местности и осуществления поиска по ней.
\end{enumerate}
